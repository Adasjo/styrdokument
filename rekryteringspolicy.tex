\documentclass{dgovdoc}

\usepackage[swedish]{babel}
\usepackage[T1]{fontenc}
\usepackage[utf8]{inputenc}

\usepackage{hyperref}

\title{Rekryteringspolicy}

\begin{document}

\maketitle

\section*{Syfte och mål}
Syftet  med  detta dokument  är att beskriva hur inval och rekrytering till såväl nämnder som 
projekt vid Datasektionen ska ske. En målsättning för Datasektionen är att alla medlemmar har lika möjligheter att engagera sig inom alla delar av organisationen. Alla medlemmar ska alltid ha möjligheten att kandidera till förtroendeposter som väljs av sektionsmötet, samt till de poster som i sin tur kan utses av förtroendevalda. Detta dokument syftar till att påvisa hur ansvariga i alla sammanhang som rör inval och rekrytering ska sträva efter att ha öppna och transparenta processer, samt hur detta kan uppnås.

\section*{Begrepp: Inval och rekrytering}
Ett \textbf{öppet val} till  Datasektionens funktionärsposter  hålls  under  sektionsmötet  där  alla medlemmar har rätt att kandidera. Undantaget är när en funktionärspost är vakantsatt, då har 
styrelsen möjligheten att tillförordna en medlem på posten tills dessa att fyllnadsval kan ske. 

Med \textbf{inval} menas att en medlem av Datasektionen väljs in till en nämnd eller ett projekt av en eller flera förtroendevalda av sektionen. Inval är exempelvis val av ledningsgrupper till projekt samt, i dagsläget, val av medlemmar i DKM och mottagningspersonal. 

\textbf{Rekrytering} syftar på hur nämnder och projekt  arbetar  för att  engagera och rekrytera fler medlemmar till sina verksamheter. 

\section*{Inval}
Inval  sker  till  nämnd  eller  projekt  ska  tydligt  marknadsföras  till  sektionens medlemmar  i sektionens  officiella  informationskanaler.  En  medlem  ska  därefter  kunna förvänta  sig  att beslut  som  tas  vid  inval  är  välgrundade  och  fattade  efter  att  ha tagit  samtliga  kandidater  i beaktning. 

Den som är ansvarig för nämnden eller projektet ska sträva efter att ge alla som söker samma förutsättningar i processen. Detta kan göras genom att samla in informationen som ligger till grund  för  invalsbeslutet  på  ett  likvärdigt  sätt  för  samtliga  kandidater.  Till exempel kan frågeformulär  eller  intervjuer  som  tar  i  beaktande  såväl  kandidaters  personliga egenskaper som tidigare erfarenheter användas.

\section*{Rekrytering}
Hur  sektionsmedlemmar  kan  deltaga  i  nämnd  eller  projekt  bör  vara  öppet  och  tydligt annonserat i sektionens officiella informationskanaler när rekrytering pågår. För de nämnder och projekt som inte utgörs av en fast grupp medlemmar bör istället aktiviteter som tar plats annonseras.

\section*{Intern rekrytering}
nom  nämnder  och  projekt  kan  det  finnas  interna  ansvarsområden  som  fördelas  mellan  de engagerade   medlemmarna.   Exempel   på   interna   ansvarsområden   kan   vara   städskri   i METAdorerna och barmästare i DKM. Möjligheten för en medlem att ansvara för ett internt ansvarsområde ska i dessa fall kommuniceras öppet inom nämnden eller projektet.

\end{document}