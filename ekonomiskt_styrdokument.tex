\documentclass{dgovdoc}

\usepackage[swedish]{babel}
\usepackage[T1]{fontenc}
\usepackage[utf8]{inputenc}

\usepackage{hyperref}

\title{Ekonomiskt styrdokument}

\begin{document}

\maketitle

\section{Bakgrund}

Detta dokument ska fylla två funktioner dels ska det ange de för sektionen
gällande regler för hur ekonomin sköts. Dokumentet ska även fungera som en
snabb utbildning i hur funktionärer och nämnder ska sköta sin ekonomi och
ekonomiska redovisning.

\section{Bokföringsskyldiga nämnder samt samtliga projekt och sektionens kassör}

Bokföringspliktiga nämnder så som DKM, mottagningen, Prylmånglaren samt projekt
sköter sin egen bokföring. Bokföring skall skötas löpande och vara färdig
senast nästföljande månadsskiftet då det ej är sommarferie. Verifikat från
respektive nämnd skall använda varsin verifikatserie enligt appendix
\ref{sec:serier}. Alla transaktioner på nämndens bankkonto ska bokföras av
respektive nämnd. Då transaktioner mellan två nämnders konton sker ska bara den
nämnd som utfört överföringen bokföra transaktionen.

\section{Skulder till engagerade}

Kvitton för inköp åt sektionen skall senast lämnas in 30 dagar efter då inköpet
gjordes för att skulden ska betalas tillbaka. Skulder ska betalas tillbaka till
engagerade inom 5 veckor (ej under ferie) under förutsättning att alla
uppgifter är rätt registrerade och det framgår i vilket syfte inköpet är gjort.

\section{Hantering av kontanter}

Mottagningen, DKM och prylmånglarn är i dagsläget de enda nämnder som har
tillräckliga rutiner för att ha en egen handkassa, andra nämnder ska därför
låta någon av dessa nämnder hantera deras kontanter och sedan lösa eventuella
skulder genom överföringar till/från respektive nämnds bankkonto. Summor över
400 kronor får inte förvaras i sektionslokalen längre än över natten och då
endast i låst utrymme.

\section{Fester}

Alla inkomster från försäljning i baren som sker under mottagningsperioden ska
tillfalla Mottagningen. Alla inkomster från försäljning i baren som inte sker
under mottagningsperioden ska tillfalla DKM. Detta på grund av att DKM och
Mottagningen är de nämnder som har en fungerande regelbunden hantering av
kontanter och det är DKMs uppgift att anordna fester. Dock kan till exempel kan
en annan nämnd komma överens med DKM eller Mottagningen om hur inkomster ska
fördelas. Detta innebär tex att dkm ansvarar för ettans fest.

\section{Kvitton}

Kvitton alternativt faktura ska skickas ut eller lämnas i någon form för all
försäljning, då även förköpsbiljetter.

\section{Fakturor}

En engagerad medlem får handla mot faktura i sektionen namn endast om det finns
en budgetpost för inköpet och personen har ett godkännande av respektive
nämndordförande eller D-rektoratet. Mottagna fakturor skall sättas in i
kvittopärmen. Om nämnden själv betalar fakturan skall detta samt fakturans
betalningsdag markeras på fakturan.

\section{Fakturering}

Nämnder ansvarar själva för att kontrollera att utskickade fakturor betalas i
tid. En kopia av fakturan skall omgående sättas in i i kvittopärmen.

\appendix

\section{Serier}
\label{sec:serier}

\begin{description}
\item[DKM] serie K
\item[dÅre] serie A
\item[Mottagningen] serie M
\item[Prylmånglaren] serie P
\item[StuDs] serie S
\item[Övriga] serie C
\end{description}

\end{document}
