\documentclass{dgovdoc}

\usepackage[swedish]{babel}
\usepackage[T1]{fontenc}
\usepackage[utf8]{inputenc}

\usepackage{hyperref}

\title{Ekonomiskt styrdokument}

\begin{document}

\maketitle

\section{Bakgrund}

Detta dokument ska fylla två funktioner. Dels ska det ange de för sektionen gällande regler för hur ekonomin
sköts. Dokumentet ska även fungera som en snabb utbildning i hur funktionärer och nämnder ska sköta sin
ekonomi och ekonomiska redovisning.

\section{Bokföringsplikt}

Bokföringspliktiga nämnder så som DKM, mottagningen, Prylmångleriet samt projekt sköter sin egen
bokföring. Bokföring skall skötas löpande och vara färdig senast nästföljande månadsskiftet då det ej är
sommarferie. Verifikat från respektive nämnd skall använda varsin verifikatserie enligt appendix \ref{sec:serier}. Alla
transaktioner på nämndens bankkonto ska bokföras av respektive nämnd. Då transaktioner mellan två
nämnders konton sker ska bara den nämnd som utfört överföringen bokföra transaktionen.

\section{Skulder till engagerade}

Kvitton för inköp åt sektionen skall senast lämnas in 30 dagar efter då inköpet gjordes för att skulden ska
betalas tillbaka. Skulder ska betalas tillbaka till engagerade inom 5 veckor (ej under ferie) under
förutsättning att alla uppgifter är rätt registrerade och det framgår i vilket syfte inköpet är gjort.

\section{Hantering av kontanter}
\label{sec:kontanter}
Mottagningen och DKM är i dagsläget de enda nämnder som har tillräckliga säkerhetsrutiner för att förvalta
större handkassor åt sektionen, andra nämnder ska därför endast ha mindre och tillfälliga handkassor.
Prylmångleriet får dock också föra en kontinuerlig handkassa. Större summor ska dock föras över till DKMs
eller Mottagningens handkassor. Överföringar mellan handkassor bokförs på samma sätt som överföringar
mellan bankkonton, det vill säga källan för överföringen är ansvarig för att bokföra den. Summor över 400
kronor får inte förvaras i sektionslokalen längre än över natten och då endast i låst utrymme. Endast
prylmångleriet äger permanent undantag att förvara högre belopp i sektionslokalen och bedömer i samråd
med D­rektoratet vilka belopp som är lämpliga att förvara där.

\section{Dryckeslager}
\subsection{Förvaltning av dryckeslager}
DKM förvaltar sektionens dryckeslager, utom under mottagningsperioden då Mottagningen tar över den
uppgiften . Det innebär att det i huvudsak är dem som fyller på lagren och bokför påfyllning. Sprit,
vin, cider, öloch läsk som serveras vid pub­ och klubbverksamhet ska lagerföras. DKM och Mottagningen avgör själva
om de vill lagerföra barkit och övriga alkoholfria produkter.

\subsection{Dryckeslager vid festtillfällen}
När en nämnd eller ett projekt har fest och använder sektionens dryckeslager så ska det bokföras av
bokföringsansvariga för den serie som nämnden eller projektet ligger under. Nämnden/projektet ska både
belastas för inköp av drycken och tilldelas försäljningen av den. Serveringsansvarig för festtillfället är ansvarig för att betalning och dagsavslut görs efter sektionens standarder.
Serveringsansvarig är även ansvarig för att handkassor hanteras enligt \ref{sec:kontanter} i detta dokument.
Serveringsansvarig ska i skälig tid innan festtillfället kontakta den nämnd som förvaltar dryckeslagren och
informera sig om gällande bestämmelser för sektionen.

\section{Kvitton}

Kvitton alternativt faktura ska skickas ut eller lämnas i någon form för all
försäljning, då även förköpsbiljetter.

\section{Fakturor}

\subsection{Inköp via faktura}
Inköp på faktura betyder för det mesta att man nyttjar ett av sektionens kreditavtal. Det går i regel bra att
nyttja kreditavtal skrivna av andra organ inom sektionen än det man handlar för, men personen som gör
inköpet ansvarar själv för att vara informerad om avtalets gränser. Stora inköp på över 50000 kr kan komma
att påverka sektionens ekonomi i stort ska meddelas kassören så fort det blir känt för sektionen. Alla fakturor
i sektionens namn ska skickas till sektionens gällande fakturaadress.

\subsection{Försäljning via faktura}
Nämnder ansvarar själva för att kontinuerligt kontrollera att utskickade fakturor betalas i tid. En kopia av
fakturan ska omgående komma bokföringsansvarig tillhanda.

\section{Budget}
Budgeten är ett instrument för att försöka förutspå framtiden och ska följas i den mån som går, men om verkligheten inte tillåter detta så måste man bortse från budgeten. I den rambudget som beslutas om på SM finns posterna: Intäkter, Utgifter, Extern kostnad och Intern kostnad. Utöver detta krävs det att det för varje nämnd på SM uppvisas en detaljbudget som överensstämmer med rambudgeten. Funktionärer som handhar en detaljbudget kan i samråd med sektionsstyrelse revidera denna under eller inför ett verksamhetsår. Sådana ändringar skall redovisas på styrelsemöte (DM).

\subsection{Förklaring}
Intäkter är vad nämnden har för totala intäkter under året. Utgifter är vad nämnden har för totala utgifter under året. Extern kostnad är de kostnader som gagnar hela sektionen, de som bedömer detta är i stigande ordning kassör, styrelse, revisorer, SM. Exempel är sittningar hela sektionen är bjuden till, inventarier till META, pubar mm. Intern kostnad är kostnader som gagnar en liten grupp inom sektionen, såsom styrelsen, DKM, andra nämnder med evenemang där alla inte är inbjudna. Exempel är mat, teambuilding, fika, fest, bungyjump, paintball, kryssning, thailandsresa mm. Det är särskilt viktigt att denna kostnad inte överstiger budgeten.

\subsection{Skyldighet}
Denna frihet som ges att få forma sin egen budget kräver fortfarande att pengar läggs på rätt saker och de med rätt att betala ut utgifter har rätt att neka utbetalningen om det bryter mot detta dokument. Beslutet kan överklagas enligt nämnda ordning i \§ 8.1.

\section{Avtal}

Avtal kan enligt svensk lag endast undertecknas av firmatecknare eller innehavare av fullmakt. Den som
skriver under ett avtal ansvarar för att orginal omgående kommer firmatecknare tillhanda. Firmatecknare
ansvarar för att förvara sektionens alla gällande avtal ordnat och samlat.

\appendix

\section{Serier}
\label{sec:serier}
Varje projekt bör bokföra under en egen serie, nämndernas serier delas upp som följer:
\begin{description}
\item[Centralt] serie C
\item[DKM] serie K
\item[Mottagningen] serie M
\item[Näringslivsgruppen] serie N
\item[Prylmångleriet] serie P
\end{description}

\end{document}
