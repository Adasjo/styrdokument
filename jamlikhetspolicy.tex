\documentclass{dgovdoc}

\usepackage[swedish]{babel}
\usepackage[T1]{fontenc}
\usepackage[utf8]{inputenc}

\usepackage{hyperref}

\title{Jämlikhetspolicy}

\begin{document}

\maketitle

\section{Allmänt}
Alla Datasektionens medlemmar har rätt att att bli behandlade likvärdigt och med respekt. Diskriminering, trakasserier, särbehandling och härskartekniker är oacceptabelt inom Datasektionen och all dess verksamhet

\section{Diskriminering och särbehandling}
Vi defnierar diskriminering och särbehandling som åtsidosättande beteende gentemot grupper eller individer, exempelvis på grund av någon av följande punkter:

\begin{itemize}
\item Könsidentitet
\item Sexuell läggning
\item Psykisk eller fysisk funktionsnedsättning
\item Etnicitet
\item Trosuppfattning och/eller livsåskådning
\item Ekonomisk och social bakgrund
\item Familjesituation
\end{itemize}

\section{Trakasserier}
Trakasserier är ett beteende som på något sätt kränker en grupp eller en individ. Alla personer som är inblandade i jämlikhetsärendenpå sektionen har rätt att bli tagna på största allvar och att bli behandlade med diskretion och respekt.

\section{Härskartekniker}
En härskarteknik är en form av social manipulation som syftar till att säkerställa en dominant grupps makt gentemot grupper med mindre makt. Denna form av maktutövande bör motarbetas och för att tillse att studiemiljön på Datasektionen förblir så öppen och välkomnande som möjligt.

\section{De fem härskarteknikerna}

\subsection{Osynliggörande}
Att tysta eller marginalisera personer genom att ignorera dem.

\subsection{Förlöjligande}
Att genom ett manipulativt sätt framställa någons argument eller person som löjlig och oviktig. Detta genomförs till exempel genom att använda slående men ovidkommande liknelser eller att inför grupp anmärka på en persons yttre.

\subsection{Undanhållande av information}
Att utestänga nåagon eller marginalisera dennes roll genom att undanhåalla den väsentlig information, till exempel genom att fatta formella beslut i informella sammanhang där inte alla berörda haft tillträde.

\subsection{Dubbelbestraffning}
Att försätta någon i en situation där personen nedvärderas och bestraffas oavsett vilket handlingsalternativ den väljer. Detta kan ske genom att beskylla någon för att vara för långsam när de genomför sina sysslor noggrant, och sedan för att vara slarvig när de försöker jobba effektivare.

\subsection{Påförande av skuld och skam}
Att få någon att skämmas för sina egenskaper, eller att antyda att något de utsätts för är deras eget fel. Detta sker ofta genom en kombination av förlöjligande och dubbelbestraffning.

\section{Datasektionen}
Alla organ inom Datasektionen skall aktivt arbeta för:

\begin{itemize}
\item Att skapa en studiemiljö där alla känner sig välkomna
\item Att motverka diskriminering och trakasserier
\item Att den egna verksamhetens jämlikhetssituation kontinuerligt diskuteras och utvärderas
\end{itemize}

\end{document}