\documentclass[a4paper,12pt]{article}

\usepackage[swedish]{babel}
\usepackage[T1]{fontenc}

\usepackage{hyperref}

\title{Reglemente}

\begin{document}

\maketitle

\section{Insignia}

\subsection{Färg}

Sektionens färg är cerise.

\subsection{Symbol}

Sektionens symbol är den grekiska bokstaven lilla delta. Deltat får icke glömmas. Utformningen och användandet av sektionens symbol regleras närmare av Informationsorganet i samråd med D-rektoratet.

\subsection{Grafisk profil}

Sektionens grafiska profil regleras av Informationsorganet i samråd med D-rektoratet.

\section{Nämnder}

Det åligger samtliga nämnder och funktionärer att följa sektionens ekonomiska styrdokument, jämlikhetspolicy, alkoholpolicy och informationsspridningsguidelines.

\subsection{Sektionslokalsgruppen}

\subsubsection{Ändamål}

Sektionslokalsgruppens syfte är att sköta, underhålla och utveckla sektionslokalen.

\subsubsection{Organisation}

Sektionslokalsgruppen leds av sektionslokalsansvariga, en från Sektionen för Medieteknik och en från Konglig Datasektionen, som väljs av respektive sektion. Beslut som fattas av de sektionslokalsansvariga måste vara enhälligt, och i de fall som sektionslokalsansvariga är oense åligger det respektive sektionsordföranden att komma överens
om och besluta i frågan. Sektionslokalsansvariga utser resten av gruppen efter ansökningar och efter att de ansökande en gång ha lagat middag åt nämnden, sig själva inkluderade. Sektionslokalsgruppen bestämmer internt organisation och planering.

\subsubsection{Verksamhet}

Sektionslokalsgruppen ansvarar för:

\begin{itemize}
  \item administration av bokning och uthyrning av sektionslokalen samt dess olika rum
\end{itemize}

\begin{itemize}
  \item tekniken, möbler/inredning, samt påfyllning av förbrukningsmaterial i sektionslokalen
\end{itemize}

\begin{itemize}
  \item utveckling av sektionslokalen
\end{itemize}

\begin{itemize}
  \item organisation av såväl nØllestädning som annan städning av sektionslokalen
\end{itemize}

\begin{itemize}
  \item att upprätta och underhålla ett dokument med gällande regler för sektionslokalen.
\end{itemize}

\subsubsection{Budget}

Sektionslokalsgruppen har en gemensam budget, som fastställs av respektive sektionsmöte. Kostnader och intäkter ska delas lika mellan sektionerna. Om enighet råder mellan sektionerna kan undantag göras från denna regel.

\subsection{Idrottsnämnden}

\subsubsection{Ändamål}

Idrottsnämnden skall verka för att sektionsmedlemmarna får tillfälle att idrotta tillsammans.

\subsubsection{Organisation}

Idrottsnämnden leds av Sektionsidrottsledaren. Övriga medlemmar är samtliga intresserade sektionsmedlemmar.

\subsubsection{Verksamhet}

Nämndens verksamhet planeras i början av varje verksamhetsår av Sektionsidrottsledaren i samråd med medlemmarna.

\subsection{Informationsorganet}

\subsubsection{Ändamål}

Informationsorganet, Ior, ansvarar för informationsspridning på sektionen och skall verka för att förbättra informationsflödet till, från och mellan sektionens medlemmar.

\subsubsection{Organisation}

Ior leds av informationsansvarige, internt benämnd Chefsåsnan. Övriga medlemmar är samtliga intresserade sektionsmedlemmar. Bland dessa utses en systemadministratör med huvudansvar för underhåll av sektionens datorresurser.

\subsubsection{Verksamhet}

Ior skall

\begin{itemize}
  \item Hjälpa D-rektoratet och övriga nämnder med informationsspridning
  \item Underhålla och utveckla sektionens webbplats
  \item Ansvara för drift och underhåll av sektionens datorresurser.
\end{itemize}

\subsection{Internationella Utskottet}

\subsubsection{Ändamål}

Internationella Utskottet, kallat DIU, skall verka för att främja internationellt utbyte på sektionen och ansvarar för mottagningen av utländska studenter.

\subsubsection{Organisation}

DIU leds av DIU:s ordförande, som till sin hjälp har DIU:S vice ordförande. Övriga medlemmar är samtliga intresserade sektionsmedlemmar.

\subsubsection{Verksamhet}

DIU skall

\begin{itemize}
  \item Samordna sektionens internationella verksamhet. Detta inkluderar att hålla kontakten med ISS ordförande på THS och de ansvariga för utbytesstudier på CSC:s kansli och institutionerna
  \item Hålla sektionsmedlemmarna informerade om internationell verksamhet på sektionen
  \item Genomföra mottagningsverksamhet för utländska studenter. Detta inkluderar att rekrytera faddrar och koordinera fadderverksamheten
  \item Fungera som kontaktperson för utländska studenter på sektionen.
\end{itemize}

\subsection{Jämlikhetsnämnden}

\subsubsection{Ändamål}

Jämlikhetsnämnden skall värna och upplysa om jämlikhet och mångfald på sektionen.

\subsubsection{Organisation}

Nämnden leds av Jämlikhetsnämndens ordförande. Styrelseledamoten för studiemiljöfrågor skall också ingå i nämnden tillsammans med övriga intresserade sektionsmedlemmar.

\subsubsection{Verksamhet}

Jämlikhetsnämnden skall

\begin{itemize}
  \item göra sektionsmedlemmarna medvetna om vad de har för rättigheter och vart de skall vända sig om de känner sig kränkta eller trakasserade
  \item arbeta för att öka jämlikheten och mångfalden på sektionen, i dess nämnder och funktionärsposter
  \item hålla minst två möten per termin där aktuella frågor gällande jämlikhet diskuteras
  \item skapa och kontinuerligt uppdatera en jämlikhetspolicy för sektionen
  \item assistera D-rektoratet med att ta fram, följa upp och vid behov revidera en handlingsplan mot trakasserier
  \item ha löpande samarbete med KTH, CSC och THS om jämlikhetsarbeten.
\end{itemize}

\subsection{Klubbmästeriet}

\subsubsection{Ändamål}

Klubbmästeriet, DKM, anordnar fester och andra sociala arrangemang för sektionens medlemmar och i vissa fall även deras eventuella vänner.

\subsubsection{Organisation}

DKM leds av klubbmästaren. Övriga medlemmar utses av DKM.

\subsubsection{Verksamhet}

Det åligger DKM att

\begin{itemize}
  \item arrangera fester, jippon, högtidliga ceremonier och andra sociala arrangemang
  \item vid behov assistera vid andra sektionsrelaterade arrangemang
  \item skicka ut inbjudningar till övriga sektioner och andra högskolor till av DKM arrangerade evenemang
  \item i sektionens informationskanaler informera om till klubbmästeriet inkomna inbjudningar till externa evenemang, samt fördela platser och biljetter om antalet aspiranter överstiger platsantalet
  \item skicka ut skriftlig inbjudan till Vårbalen till medlemmar av sektionens ordnar.
\end{itemize}

\subsubsection{Bokföringsplikt}

DKM är bokföringspliktigt.

\subsection{Mottagningen}

\subsubsection{Ändamål}

Mottagningen har som syfte att ta hand om och roa de nyantagna på till sektionen hörande program, främst på grundnivå men i tillämplig utsträckning även på avancerad nivå, och få dem att lära känna varandra och äldre sektionsmedlemmar. Vidare syftar Mottagningen till att lära nØllan hur KTH, THS och sektionen fungerar och är uppbyggda utifrån ett studentperspektiv. Mottagningen skall även sträva efter att ge nØllan bästa möjliga introduktion till deras studier, till sektionen och till studentliv i allmänhet.

\subsubsection{Organisation}

Mottagningen som helhet leds gemensamt av Konglig Öfverdrif och Storasyskonet, hädanefter benämnda presidiet. Presidiet ansvarar inför D-rektoratet för Mottagningens verksamhet och ekonomi.

Mottagningen består av två huvudsakliga grenar

Det Kongliga Drifveriet
Det Kongliga Dadderiet

Drifveriet leds av Konglig Öfverdrif. Dadderiet leds av Storasyskonet. Presidiet väljer ut resten av mottagningens ledningsgrupp, Titel.

Titeldrifveriet utser gemensamt övriga medlemmar av Drifveriet.

Titeldadderiet utser gemensamt övriga medlemmar av Dadderiet.

Samråd skall ske inom hela Titel när Mottagningens sammansättning beslutas. Utöver vad som ovan angivits får presidiet besluta om eventuella undergrupper till de olika grenarna, externa arbetsgrupper eller i övrigt andra befattningar knutna till Mottagningen, samt hur dessa i så fall skall utses.

\subsubsection{Rekrytering till Mottagningen}

Titelgruppen skall utlysa platserna i höstens mottagning under våren. Alla sökande skall erbjudas intervjuer innan mottagningens sammansättning fastslås. De sökandes egna prioriteringar av sökta uppdrag inom Mottagningen skall respekteras.

\subsubsection{Sektionsordförandes roll}

Sektionsordförande är genom KTH:s och THS regler för Mottagningen ytterst ansvarig för denna. Presidiet skall därför löpande hålla sektionsordförande informerad om verksamheten och samråda med denne i frågor av principiell vikt. Sektionsordförande skall å sin sida fungera som stöd åt presidiet och bistå dem i deras ledningsfunktion.

Sektionsordförande äger alltid rätt att fatta de beslut och vidta de åtgärder som denne finner vara nödvändiga för att KTH:s och THS regler för Mottagningen skall upprätthållas. I den mån det är möjligt skall dock samråd alltid ske med presidiet inför ett sådant beslut eller åtgärd.

Sektionsordförande bör inte inneha något annat uppdrag inom Mottagningen än det som följer av ordförandeskapet.

\subsubsection{Verksamhet}

Mottagningen ansvarar för att aktiviteter som främjar Mottagningens ändamål anordnas. Dessa aktiviteter avslutas sedan med nØlleGasquen och nØllans eventuella upphöjelse till Ettan. För planeringen av verksamheten ansvarar Titelgruppen.

\subsubsection{Bokföringsplikt}

Mottagningen är bokföringspliktig.

\subsection{Näringslivsgruppen}

\subsubsection{Ändamål}

Näringslivsgruppen har till uppgift att informera näringslivet om sektionen och datateknikprogrammet, att främja sektionsmedlemmarnas status på arbetsmarknaden samt att inbringa sponsorpengar till sektionen.

\subsubsection{Organisation}

Näringslivsgruppen leds av Näringslivsansvarig. Övriga medlemmar är samtliga intresserade sektionsmedlemmar.

\subsubsection{Verksamhet}

Näringslivsgruppen skall

\begin{itemize}
  \item göra reklam för sektionen och datateknikprogrammet
  \item samordna de av sektionens verksamheter som riktar sig mot näringslivet, så att företagen bemöts på ett professionellt sätt
  \item hålla och ständigt förbättra kontakten med näringslivet
  \item sköta Näringslivsgruppen faktureringar
  \item se till att sektionen uppfyller avtal framförhandlade av Näringslivsgruppen.
\end{itemize}

\subsubsection{Avtal}

Näringslivsansvarig har rätt att förhandla fram avtal med företag och organisationer, men dessa måste sedan undertecknas av firmatecknare.

\subsection{Qulturnämnden}

\subsubsection{Ändamål}

Qulturnämnden, QN, skall genomföra kulturella evenemang och tillhandahålla ett rikt kulturutbud för sektionens medlemmar.

\subsubsection{Organisation}

QN leds av Qulturattachén. Övriga medlemmar är intresserade sektionsmedlemmar som har jobbat på minst ett QN-arrangemang.

\subsubsection{Verksamhet}

QN bör

\begin{itemize}
  \item anordna regelbundna aktiviteter som höjer den kulturella nivån på sektionen, såsom filmvisningar, spelkvällar och syjuntor
  \item tillhandahålla böcker, spel och annat kulturellt i sektionslokalen
  \item bedriva godis- och läskförsäljning i sektionslokalen
  \item anordna arrangemang utanför sektionslokalen, såsom besök på utställningar, opera eller annan kulturellt relaterad verksamhet
  \item årligen utse och Q-märka en person, ett ting, en företeelse eller något annat som man anser är ett gott exempel på god qultur. Q-märkningen skall förevigas exempelvis i form av ett tygmärke.
\end{itemize}

\subsection{Redaqtionen}

\subsubsection{Ändamål}

Redaqtionen ansvarar för att sektionens tidning dBuggen utkommer med ett lagom antal nummer varje läsår, samt att nØlledBuggen utkommer med exakt ett nummer inför Mottagningen varje läsår.

\subsubsection{Organisation}

Redaqtionens arbete leds av Chefredaqteuren, Che Fred. Övriga medlemmar är samtliga intresserade sektionsmedlemmar.

\subsubsection{Verksamhet}

\paragraph{dBuggen}

dBuggens innehåll och inriktning bestäms av sektionens medlemmar genom deras val av Chefredaqteur. Av tradition utkommer exakt ett nummer av dBuggen klockan kvart i fem.

\paragraph{nØlledBuggen}

nØlledBuggens innehåll skall bestå av gammalt material som tidigare publicerats i dBuggen, information rörande Mottagningen, allmän information om sektionens olika delar, samt övrigt material som Chefredaqteuren finner lämpligt.

\paragraph{Annonser}

Alla världsliga detaljer rörande försäljning av annonser i dBuggen och nØlledBuggen sköts av Näringslivsgruppen. Redaqtionen ansvarar endast för att införa beställda annonser och ge dessa en studentikos framtoning.

\paragraph{Pris}

Priset för ett exemplar av dBuggen eller nØlledBuggen skall vara noll prisbasbelopp.

\subsection{Spexmästeriet}

\subsubsection{Ändamål}

Spexmästeriets ändamål är att bringa glädje till sektionens medlemmar genom att ägna sig åt spex, gyckel, musik, dans och andra sceniska underhållningsformer. Verkar för att sektionens spex Dataspelet skall uppföras med jämna mellanrum.

\subsubsection{Organisation}

Spexmästeriet leds av Spexmästaren, som samordnar all aktivitet inom nämnden och är huvudansvarig för medverkar i Sångartäflan. För Spexmästeriets ekonomi ansvarar Spexmästeriets kassor, internt benämnd Debiteur. För Dataspelet ansvarar Spexdireqteuren. Till sin hjälp i arbetet med att leda Dataspelet har Spexdireqteuren Direqtionen, som utgörs av denne själv, Spexmästaren, Debiteuren och chefen för Näringslivsgruppen. Övriga medlemmar är intresserade människor, både teknologer och utomstående. Alla medlemmar tillhör en eller flera av följande grupper. Varje grupp utser inom sig en chef, med undantag för skådisgruppen, som leds av Regisseuren.

\begin{itemize}
  \item Dansgruppen
\end{itemize}

Skapar dynamik på scenen genom att dansa till den musik orkestern (eller någon annan lämplig ljudkälla) bidrar med.

\begin{itemize}
  \item Dekor- \& rekvisitagruppen
\end{itemize}

Bygger, målar och skapar på andra sätt kulisser, prylar och allt annat som skall finnas på scen.

\begin{itemize}
  \item Ljud- \& ljusgruppen
\end{itemize}

Bidrar till helhetsintrycket med bländande tekniska effekter (och ibland helt vanlig belysning och ljudförstärkning).

\begin{itemize}
  \item Orkestergruppen
\end{itemize}

Spelar instrument av alla de slag, antingen på egen hand eller ackompanjerandes scengruppen.

\begin{itemize}
  \item PR-gruppen
\end{itemize}

Gör reklam för Spexmästeriets olika evenemang, samt sköter sponsorkontakter.

\begin{itemize}
  \item Scengruppen
\end{itemize}

Ägnar sig åt skådespeleri, sång och allt annat sådant som man gör på scen (även om dans huvudsakligen förekommer i dansgruppen).

\begin{itemize}
  \item Skrivargruppen
\end{itemize}

Skriver det material som scengruppen framför --- manus, låttexter osv.

\begin{itemize}
  \item Skådisgruppen (Dataspelet)
\end{itemize}

Den delmängd av scengruppen som innehar roller i den pågående uppsättningen på Dataspelet.

\begin{itemize}
  \item Sy- \& sminkgruppen
\end{itemize}

Förser scen- och dansgruppen.

\begin{itemize}
  \item Trivselgruppen
\end{itemize}

Fixar fester, pausfika, filmkvällar och annat myspys för såväl resten av Spexmästeriet som de som gästar Spexmästeriets evenemang.

Utöver dessa gruppchefer skall Spexmästaren, eller för Dataspelet Spexdireqteuren, vid behov utse särskilda befattningshavare, såsom regisseur, maestro, koreograf eller scenograf.

Gruppcheferna, övriga befattningshavare samt medlemmar i Dataspelets skådisgrupp bör vara eller ha varit medlemmar i sektionen.

\subsubsection{Verksamhet}

Spexmästeriet skall sträva efter att varje år delta i Sångartäflan och att med jämna mellanrum, gärna vartannat år, uppföra sektionens spex Dataspelet. Däremellan skall spexmästeriet främja spex- och gyckelkulturen genom att till exempel uppföra gyckel på fester och arrangemang, särskilt under Mottagningen.

\subsubsection{Bokföringsplikt}

Spexmästeriet är bokföringspliktigt för all verksamhet relaterad till Dataspelet. Nämndens verksamhet i övrigt omfattas av sektionens centrala bokföring, såvida inte D-rektoratet i samråd med Spexmästaren beslutar annorlunda för givet räkenskapsår.

\subsection{Studienämnden}

\subsubsection{Ändamål}

Studienämndens syfte är att bevaka och förbättra utbildningskvaliteten och studiemiljön för sektionens medlemmar på kort såväl som lång sikt.

\subsubsection{Organisation}

Studienämnden leds av Studienämndens ordförande. Övriga medlemmar är

\begin{itemize}
  \item Styrelseledamot för utbildningsfrågor
  \item Styrelseledamot för studiemiljöfrågor
  \item Programansvarig student
  \item sektionens teknologstudievägledare på CSC:s kansli
  \item årskursrepresentanter
  \item övriga intresserade sektionsmedlemmar.
\end{itemize}

Studienämndens ordförande är ansvarig för rekryteringen av årskursrepresentanter och studentrepresentanter när vakanta platser finns.

\subsubsection{Verksamhet}

Studienämnden skall hålla möte minst en gång per period. Mötena skall vara för samtliga intresserade sektionsmedlemmar.

Den huvudsakliga verksamheten skall vara att

\begin{itemize}
  \item inhämta teknologernas åsikter om studiesituationen och med dessa som utgångspunkt arbeta för att förbättra kvaliteten på utbildningen
  \item utvärdera information och beslut från KTH:s organ
  \item bevaka och söka förbättra den fysiska och psykosociala studiemiljön.
\end{itemize}

Årskursrepresentanternas främsta uppgift är att samordna utvärderingen av kurserna i respektive årskurs. Kursutvärderingarna skall vidarebefordras till kursansvarig, examininator, CSC:s kansli samt övriga personer som Studienämnden anser vara berörda.

Studentrepresentanternas främsta uppgift är att bevaka verksamheten i KTH:s organ, att vidareförmedla information till Studienämnden samt att sträva efter att påverka KTH:s beslut i linje med studenternas åsikter.

Studienämndens dokument skall i så stor utsträckning som möjligt finnas tillgängliga i elektronisk form.

\subsection{Konglig Östrogennämnden}

\subsubsection{Ändamål}

Nämndens syfte är att främja tjejers intressen på Datasektionen.

\subsubsection{Organisation}

Ordförande för Konglig Östrogennämnden är Öfvermatronan, som väljs på SM.

\subsubsection{Verksamhet}

Konglig Östrogennämnden ska verka för att ge tjejer på datasektionen en möjlighet att
nätverka med varandra. Nämnden ska anordna middag för alla tjejer två gånger per
termin samt anordna en tjejgasque en gång per år. Nämnden ska medverka under
mottagningen för att få fler tjejer att känna sig välkomna och även anordna
företags/inspirations-events under året. Nämnden ska varje år dessutom utse en
hedersdam.

\subsection{Konglig Fenixorden}

\subsubsection{Eventualitet}

Då det ligger i fenixordens natur att ibland upplösas gäller övriga paragrafer endast då Konglig Fenixorden är instiftad av sektionsmötet.

\subsubsection{Ändamål}

Att en gång varje år försöka uppnå upplösning av nämnden, och att därefter verka för nämndes samtidiga återinstiftande.

\subsubsection{Organisation}

Ordförande för Fenixorden väljs på Glögg-SM. Övriga medlemmar är intresserade sektionsmedlemmar.

\subsubsection{Verksamhet}

Fenixorden ska under varje verksamhetsperiod verka för att det vid ett passande högtidligt tillfälle delas ut en medalj ``Fenixorden" till en förtjänt sektionsmedlem.

\section{Funktionärer}

\subsection{D-rektoratet}

\subsubsection{Sektionsordförande}

Är ordförande för sektionen och leder D-rektoratets arbete. Ansvarar, tillsammans med ledamoten för sociala frågor och relationer, för kontakten med THS och övriga sektioner. Ansvarar för att utveckla Team-CSC, ett samarbete med Sektionen för Medieteknik. Ordförande bör ej åta sig andra tidskrävande uppdrag under sin mandatperiod.

\subsubsection{Vice sektionsordförande}

Hjälper sektionsordförande med de löpande uppgifterna. Har ansvar för kontakterna med sektionens tillfälliga nämnder och projekt som saknar annan kontaktperson inom styrelsen.

\subsubsection{Sekreterare}

Är D-rektoratets sekreterare. Hämtar och delar ut sektionens post. Ansvarar för att protokoll från DM och SM anslås i enlighet med stadgarna. Har ansvar för kontakterna med ordföranden för Informationsorganet, Valberedningen och Redaqtionen samt Sektionshistorikern och sektionens ledamöter och suppleanter i Kårfullmäktige.

\subsubsection{Kassör}

Ansvarar för sektionens ekonomi. Planerar budget, sköter löpande bokföring samt placerar sektionens tillgångar. Bistår med sektionens ekonomiska beslutsunderlag. Har ansvar för kontakterna med Näringslivsansvarig och Prylmånglaren. Kassören ansvarar även för att det finns ett uppdaterat styrdokument för sektionens ekonomi.

\subsubsection{Ledamot för sociala frågor och relationer}

Har tillsammans med sektionsordförande ansvar för kontakterna med THS och andra sektioner, samt skall arbeta för att främja kontakterna med andra kårer inom och utom Sverige. Har ansvar för kontakterna med fanbärare, vice fanbärare, ordförande för Klubbmästeriet, Konglig Östrogennämnden, Sektionslokalsgruppen, Spexmästeriet, Qulturnämnden, Mottagningens funktionärer och Fenixorden samt sektionsprojekten dÅre, METAspexet och vårbalen, i mån av existens. Väljs på Val-SM. Har läsår som mandatperiod.

\subsubsection{Ledamot för studiemiljöfrågor}

Skall verka för en bättre studiemiljö på sektionen och fungera som sektionens
studerandeskyddsombud. Har ansvar för kontakterna med ordförande för Internationella Utskottet, Jämlikhetsnämnden och Idrottsnämnden. Väljs på Val-SM. Har läsår som mandatperiod.

\paragraph{Studerandeskyddsombud}

I rollen som studerandeskyddsombud skall ledamoten såväl proaktivt som reaktivt verka för att sektionsmedlemmarnas studiemiljö är så bra som möjligt. Detta görs bland annat genom att

\begin{itemize}
  \item ta emot och behandla anmälningar rörande studiemiljön från sektionsmedlemmar
  \item agera som informationskanal mellan sektionsmedlemmarna och KTH samt THS i arbetsmiljöfrågor
  \item närvara på skyddsronder i lokaler där sektionsmedlemmarna ofta vistas.
\end{itemize}

\subsubsection{Ledamot för utbildningsfrågor}

Skall verka för en bättre utbildningskvalitet och utbildningsmiljö på sektionen. Har ansvar för kontakterna med ordförande för Studienämnden och Programansvarig student samt sektionsprojektet STUDS, i mån av existens. Väljs på Val-SM. Har läsår som mandatperiod.

\subsection{Nämndordförande}

\subsubsection{Chefredaqteur}

Är ordförande för Redaqtionen. Ser till att ett lagom antal vanliga nummer av dBuggen, plus exakt ett nummer av nØlledBuggen utkommer varje år.

\subsubsection{Informationsansvarig}

Är ordförande för Informationsorganet. Väljs på Val-SM. Har läsår som mandatperiod.

\subsubsection{Internationella Utskottets ordförande}

Främjar internationellt utbyte på sektionen.

\subsubsection{Jämlikhetsnämndens ordförande}

Är ordförande för Jämlikhetsnämnden. Väljs på Val-SM. Har läsår som mandatperiod.

\subsubsection{Klubbmästare}

Är ordförande för Klubbmästeriet. Väljs på Val-SM. Har läsår som mandatperiod.

\subsubsection{Konglig Lokalchef}

Konglig lokalchef är sektionslokalsansvarig och leder sektionslokalsgruppen tillsammmans med motsvarande post vid Sektionen för Medieteknik. Väljs på Val-SM. Har läsår som mandatperiod.

\subsubsection{Konglig Öfverdrif}

Är tillsammans med Storasyskon ansvarig för Mottagningen.

\subsubsection{Näringslivsansvarig}

Är ordförande för Näringslivsgruppen. Ansvarar för kontakter med näringslivet.

\subsubsection{Qulturattaché}

Är ordförande för Qulturnämnden. Väljs på Val-SM. Har läsår som mandatperiod.

\subsubsection{Sektionsidrottsledare}

Är ordförande för Idrottsnämnden. Väljs på Val-SM. Har läsår som mandatperiod.

\subsubsection{Spexmästare}

Är ordförande för Spexmästeriet. Samordnar all spexrelaterad verksamhet på sektionen.

\subsubsection{Storasyskon}

Är tillsammans med Konglig Öfverdrif ansvarig för Mottagningen.

\subsubsection{Studienämndens ordförande}

Är ordförande för Studienämnden. Ansvarar för utbildningsbevakning. Väljs på Val-SM. Har läsår som mandatperiod. Är ersättare till studerandeskyddsombudet.

\subsubsection{Öfvermatrona}

Är ordförande för Konglig Östrogennämnden. Väljs på Val-SM. Har läsår som
mandatperiod.

\subsection{Övriga funktionärer}

\subsubsection{Fanbärare}

Fanbärarna försvarar sektionens ära genom att bära dess fana vid olika högtidliga tillfällen. Observera att fanan skall hållas högt.

Att vara Fanbärare är en mycket hedersfylld post på sektionen.

Fanbärarna skall närvara på så många som möjligt av de tillställningar, till vilka de inbjuds av THS, samt i andra sammanhang efter beslut av D-rektoratet.

Sektionen bekostar fanbärarnas alkoholfria deltagaravgift för de evenemang där fanan bärs.

Fanbärarna bär huvudansvaret för att sektionens fana hålls i gott skick. Väljs på Val-SM. Har läsår som mandatperiod.

\subsubsection{Vice fanbärare}

Vice fanbärare försvarar sektionens ära när ordinarie fanbärare ej har möjlighet att göra det. Vid arrangemang med begränsat deltagarantal har fanbäraren företräde framför vice fanbäraren. Väljs på Val-SM. Har läsår som mandatperiod.

\subsubsection{Kårfullmäktigeledamöter}

\paragraph{Ändamål}

Kårfullmäktigeledamöterna och -suppleanterna representerar sektionen i THS Kårfullmäktige.

\paragraph{Organisation}

Sektionen har en ordinarie ledamot och en suppleant för varje mandat i THS Kårfullmäktige som sektionen tilldelats.

Ledamöterna och suppleanterna utser inom sig en samordnare som har till uppgift att skicka anmälningar till talmanspresidiet och i övrigt leda ledamöternas arbete.

\paragraph{Verksamhet}

Såväl ledamöter som suppleanter skall delta på så många sammanträden av THS Kårfullmäktige som möjligt. De är solidariskt ansvariga för att sektionen är fulltalig vid samtliga KF.

Ledamöter och injusterade suppleanter skall rösta så, som de själva finner lämpligast. Dock skall de i möjligaste mån bevaka sektionens och sektionsmedlemmarnas intressen. I frågor av stor betydelse eller där väldigt delade meningar råder kan de välja att hänskjuta beslutet om hur de skall rösta till SM. Samtliga ordinarie ledamöter och suppleanter frånsett sektionsordförande och vice sektionsordförande har läsår som mandatperiod.

\subsubsection{Programansvarig student}

Arbetar tillsammans med Studienämndens ordförande för att förbättra utbildningens kvalitet. Väljs på Val-SM. Har läsår som mandatperiod.

\subsubsection{Prylmånglaren}

Prylmånglaren har som syfte att förse sektionen med sektionsrelaterade prylar, såsom märken, pins, medaljer, spegater och sångböcker. Väljs på Glögg-SM. Har kalenderår som mandatperiod.

\paragraph{Verksamhet}

Prylmånglaren skall kontinuerligt se till att det finns sektionsmärken, pins, spegater, sångböcker och dylikt till hands. Om något skulle ta slut skall det beställas nya omedelbart.

Prylmånglaren skall sträva efter att finnas till hands inför större fester och vid förfrågan. Prylmånglaren skall dessutom sträva efter att finnas till hands inför och under Mottagningen.

Prylmånglaren skall inventera alla prylar varje månad för att förenkla inför bokföringen.

Prylmånglaren skall under Mottagningen göra sig synlig bland nØllan och arrangera ett tillfälle då nØllan får prova overaller. Prylmånglaren har som ansvar att överse införskaffandet av ettans overaller. Prylmånglaren skall också hjälpa ettan att utforma och beställa årskursmärken.

Vid jubileum och andra större händelser på sektionen bör Prylmånglaren i samarbete med ansvariga för händelsen utforma och beställa märken och andra prylar relaterade till händelsen.

\subsubsection{Revisorer}

\paragraph{Ändamål}

Revisorernas uppgift är att övervaka D-rektoratet och nämndernas arbete tillsammans med THS revisorer.

\paragraph{Organisation}

Enligt sektionens stadgar finns två revisorer, utsedda av SM. De skall

\begin{itemize}
  \item övervaka D-rektoratet och nämndernas arbete i sektionens namn,
  \item övervaka den löpande bokföringen och, om så anses behövas, kräva att en delårsrapport presenteras,
  \item revidera ekonomisk bokföring från sektionens organ,
  \item övervaka upprättandet av verksamhetsberättelsen för sektionen, samt
  \item vara skiljemän vid tvister inom sektionen där parterna inte behöver använda sig av SM eller THS styrelse, revisorsgrupp eller Kårfullmäktige.
\end{itemize}

Tvister där sektionens revisorer inte kan vara skiljemän inkluderar, men är inte begränsat till, tvister där revisorerna kan anses jäviga.

\paragraph{Verksamhet}

Revisorerna för ett verksamhetsår är ålagda att revidera samtliga av sektionens verksamheter för det året, samt att i samråd med tidigare och senare revisorer revidera löpande verksamhet som löper över flera år. Det åligger de senast valda revisorerna att ansvara för att revisionerna genomförs.

\subparagraph{Revisionsberättelse och -rapport}

Varje revision dokumenteras i två skrivelser. Av dessa två är revisionsberättelsen offentlig.

Revisionsrapporten är en detaljerad beskrivning av anmärkningar i bokföring och/eller verksamhet. Den ligger till grund för kommunikationen mellan sektionens revisorer samt mellan sektionens och THS revisorer. I revisionsrapporten bör antecknas

\begin{itemize}
  \item anmärkningar på bokföringens genomförande och strukturering,
  \item händelser i verksamheten som påverkat andra organ av sektionen, samt
  \item revisorernas uppfattning om verksamheten givet verksamhetsberättelse och samtal med av SM utnämnda nämndansvariga.
\end{itemize}

Revisionsberättelsen baseras på revisionsrapporten och är det dokument som presenteras för SM vid fråga angående ansvarsfrihet. Revisionsberättelsen är en kort sammanfattning av rapporten, med avslutande rekommendation att tillstyrka eller avstyrka beviljande av ansvarsfrihet. Rekommendationen kan utelämnas då särskilda skäl föreligger det emot.

\subparagraph{SM}

Vid ett SM där en revisionsberättelse skall läsas, kan revisorerna, enligt föregående avsnitt, ge en rekommendation till SM angående beviljande av ansvarsfrihet. SM bör beakta revisorernas samlade arbete vid efterföljande omröstning.

Innan fråga angående ansvarsfrihet tas upp på SM skall revisorerna ansvara för att de berörda ekonomiskt ansvariga inbjuds till SM.

\subparagraph{Normativa rekommendationer}

De rekommendationer som ges nedan bör följas för att förenkla och accelerera revisionsförfarandet.

\subparagraph{Verksamhetsberättelse}

Det åligger sektionsordförande att ansvara för att en verksamhetsberättelse (VB) uppförs efter (eller i samband med) avslutat verksamhetsår. Denna VB skall (som ett minimum) innehålla en berättelse från varje ordförande för bokföringspliktig nämnd samt bokföringspliktiga funktionärer. VB skall vara revisorerna tillhanda innan första SM på nästkommande verksamhetsår.

\subparagraph{Bokföring}

Det åligger de ekonomiskt ansvariga i varje bokföringspliktig nämnd att lämna en avslutad bokföring till revisorerna. Bokföringen skall vara revisorerna tillhanda innan första SM på nästkommande verksamhetsår, om inte starka skäl föreligger däremot.

Det åligger även de ekonomiskt ansvariga att på ett professionellt och strukturerat sätt inventera lager och kassa vid överlämnandet till nästa förtroendevald på posterna. Överlämningsdokumentet skall finnas revisorerna tillhanda tillsammans med bokföringen.

\paragraph{Mandatperiod}

Revisorn väljs till sakrevisor för sektionen under ett verksamhetsår samt till funktionärsposten revisor under perioden 1 januari till 30 juni nästkommande år.

\subsubsection{Sektionshistoriker}

Sektionshistorikern skall se till att sektionens ärorika historia inte faller i
glömska, dels genom att samla in historisk information och historiska föremål och dels genom att föra sagda information vidare till och visa upp sagda föremål för sektionsmedlemmarna i lämpliga sammanhang. Sektionshistorikern ansvarar även för sektionens alumniverksamhet.

Sektionshistorikern avgör själv hur han bäst uppfyller ändamålet. Sektionshistorikern arbetar ensam, men har som kunskapskälla tillgång till GUDAR-gruppen, Gamla Uvar på Data med Anrika Redogörelser. Väljs på Val-SM. Har läsår som mandatperiod.

\subsubsection{Valberedaren}

\paragraph{Ändamål}

Valberedaren har till uppgift att administrera de val som genomförs vid sektionen samt nomineras till THS Kårfullmäktiges valnämnd.

\paragraph{Verksamhet}

Valberedaren kan välja att arbeta själv eller att ta hjälp av valfritt antal personer. Dessa personer utgör i så fall valberedningen. Väljs på Val-SM. Har läsår som mandatperiod.

Valberedaren skall genomföra intervjuer med alla kandidater.
Resultatet av intervjuerna skall göras tillgängligt för sektionens medlemmar på lämpligt sätt
senast samma dag som en nominering senast skall bekräftas enligt sektionens stadgar \S5.2.4.

Nominering till funktionärspost skall delgivas kandidaten senast 3 läsdagar före det SM där valet sker.

\paragraph{Skyldigheter}

Valberedaren ansvarar för att alla val utlyses i enlighet med sektionens stadgar. Därmed skall valberedaren samla in nomineringar samt tillfråga de nominerade i god tid innan SM.

Valberedaren är även ansvarig för att valen av sektionsordförande, vice sektionsordförande och kårfullmäktigeledamöter och suppleanter genomförs i enlighet med THS och sektionens stadgar. Valberedaren skall ansvara för att personbeskrivningar inför urnval arkiveras och delges sektionshistorikern.

\paragraph{Urnval}

Vid urnval ska valurnan hållas tillgänglig för sektionens medlemmar i sektionslokalen eller annan likvärdig plats under minst fem läsdagar. Urnan skall finnas tillgänglig åtminstone en timme per dag, i första hand under lunchtid.

\section{Externa representanter}

\subsection{Representation i råd på THS}

Samtliga ordinarie ledamöter skall om möjligt delta på varje möte. Ordinarie ledamots kontaktperson i D-rektoratet träder in som suppleant om ingen ordinarie ledamot kan närvara, och ingen annan överenskommelse skett mellan de bägge. I råd med öppet medlemskap får dock suppleanter naturligtvis delta på alla möten.

\subsubsection{Arbetsmarknadsrådet}

\paragraph{Ordinarie}

Näringslivsansvarig

\subsubsection{Idrottsrådet}

\paragraph{Ordinarie}

Sektionsidrottsledare

\subsubsection{Informationsrådet}

\paragraph{Ordinarie}

Informationsansvarig

\subsubsection{Internationella rådet}

\paragraph{Ordinarie}

Internationella Utskottets ordförande

\subsubsection{Jämlikhetsrådet}

\paragraph{Ordinarie}

Jämlikhetsnämndens ordförande

\subsubsection{Pubmästarrådet}

\paragraph{Ordinarie}

Klubbmästaren

\subsubsection{Mottagningsrådet}

\paragraph{Ordinarie}

Mottagningens titelgrupp. Om THS styrelse så bestämmer är även sektionsordförande ordinarie ledamot.

\paragraph{Suppleant}

Styrelseledamot för studiemiljöfrågor

\subsubsection{Ordföranderådet}

\paragraph{Ordinarie}

Sektionsordförande

\paragraph{Suppleant}

Vice sektionsordförande

\subsubsection{Redaktionsrådet}

\paragraph{Ordinarie}

Chefredaqteur

\paragraph{Suppleant}

Sekreterare

\subsubsection{Studiemiljörådet}

\paragraph{Ordinarie}

Styrelseledamot för studiemiljöfrågor

\paragraph{Suppleant}

Styrelseledamot för utbildningsfrågor

\subsubsection{Utbildningsrådet}

\paragraph{Ordinarie}

Studienämndens ordförande, Programansvarig student

\paragraph{Suppleant}

Styrelseledamot för utbildningsfrågor

\subsection{Representation inom organ på KTH}

\subsubsection{Val av representanter}

Representanter till många av dessa organ väljs inte direkt av sektionen, utan nomineras till THS styrelse som sedan tillsätter posterna.

\subsubsection{Förteckning}

\paragraph{Skolstyrelsen}

Fattar beslut om bland annat budget och bokslut, samt de frågor som dekanen anser att styrelsen skall besluta om. Representanten ansvarar även för att kalla till skolråd. Skolrådet är till för att samordna sektionerna vid skolan i deras arbete att påverka skolan.

Till skolstyrelsen skall en ordinarie och en suppleant nomineras av D-rektoratet.

\paragraph{Ledningsgruppen}

Bereder och lämnar förslag inför viktigare beslut som skall fattas på skolan.

Till denna grupp finns en plats som delas med Media. Det är lämpligt att D-rektoratet nominerar en representant.

\paragraph{AHA}

Administration-Handläggning-Assistans, en grupp med syfte att se till att skolans administrativa resurser utnyttjas på bästa sätt.

Det är lämpligt att D-rektoratet väljer en representant till denna grupp.

\paragraph{Arbetsmiljögruppen}

Arbetar för en bättre arbetsmiljö på skolan.

I denna grupp sitter styrelseledamot för studiemiljöfrågor.

\paragraph{Grundutbildningsgruppen}

Arbetar för förbättring av grundutbildningen.

I denna grupp sitter Programansvarig student som ordinarie, samt Studienämndens ordförande som suppleant.

\paragraph{Lärarsystemgruppteknologsamverkansgruppen}

Verkar för att förbättra samverkan mellan lärare, Systemgruppen och teknologer på skolan.

Det är lämpligt att D-rektoratet väljer en representant till denna grupp.

\subparagraph{Namnet}

Ja, den heter faktiskt så.

\paragraph{Jämlikhet, mångfald och likabehandlingsgruppen}

Arbetar för att förbättra jämlikheten och mångfalden på skolan.

I denna grupp sitter Jämlikhetsnämndens ordförande.

\paragraph{PAGOD}

Arbetar för att förbättra datormiljön på CSC.

Det är lämpligt att D-rektoratet väljer en representant till denna grupp.

\paragraph{Tjänsteförslagsnämnden}

Tjänsteförslagsnämnden har till uppgift att bereda och avge förslag beträffande vissa anställningar.

Till denna nämnd finns en plats som delas mellan CSC, ITC och EES. Det är lämpligt att D-rektoratet nominerar en representant.

\section{Ordinarie SM}

\subsection{Förteckning}

\subsubsection{Budget-SM}

Ett SM skall hållas på hösten senast 15 november och benämnas Budget-SM. Budget-SM skall speciellt behandla frågan om budget för nästkommande verksamhetsår.

\subsubsection{Glögg-SM}

Ett SM skall hållas i december och benämnas Glögg-SM.

Mötesordföranden skall på Glögg-SM bära cerise tomteluva. Detta för att försäkra sig om att ingen sektionsmedlem blir sittande i Cerise eller motsvarande terminalinrättning på julafton.

\subsubsection{Revisions-SM}

Ett SM skall hållas på våren senast 31 mars och benämnas Revisions-SM. Revisions-SM skall speciellt granska D-rektoratets, nämndernas och funktionärernas berättelser samt frågan om ansvarsfrihet för D-rektoratet och nämnder med bokföringsplikt.

\subsubsection{Val-SM}

Ett SM skall hållas efter Revisions-SM senast 15 maj och benämnas Val-SM.

\subsection{Mall för föredragningslista}

Följande är förteckning över rekommenderad föredragningslista för ett Ordinarie SM.

\begin{enumerate}
  \item Mötets högtidliga öppnande
  \item Formalia
  \begin{enumerate}
      \item Val av mötesordförande
      \item Val av mötessekreterare
      \item Val av två justeringspersoner tillika rösträknare
      \item Mötets behöriga utlysande
      \item Mötesordförandes reglementesenliga klädedräkt (Generellt endast på Glögg-SM)
      \item Eventuella adjungeringar
      \item Anmälan av övriga frågor
      \item Fastställande av föredragningslistan
      \item Tidigare mötens protokoll
  \end{enumerate}
  \item Rapporter
  \begin{enumerate}
      \item D-rektoratet
      \begin{enumerate}
          \item Presidiet
          \item Kassör
          \item Sekreterare
          \item Ledamot för sociala frågor och relationer
          \item Ledamot för studiemiljöfrågor
          \item Ledamot för utbildningsfrågor
      \end{enumerate}
      \item Övriga funktionärer
      \begin{enumerate}
          \item DIU
          \item DKM
          \item Sektionslokalsgruppen
          \item Fanbärare
          \item Ior
          \item Jämlikhetsnämnden
          \item KF-ledamöter
          \item Mottagningen
          \item Näringslivsgruppen
          \item Programansvarig student
          \item Prylmånglaren
          \item QN
          \item Redaqtionen
          \item Revisorer
          \item SIL
          \item Sektionshistoriker
          \item Spexmästeriet
          \item Studienämnden
          \item Valberedaren
      \end{enumerate}
      \item Kåren
  \end{enumerate}
  \item Bordlagda ärenden
  \item Andra läsningen
  \item Revisionsärenden (Generellt endast på Revisions-SM)
  \item Beslutsärenden (Budgetärenden och dylikt)
  \item Fyllnadsval (Bordlagda valärenden hanteras under ``Bordlagda ärenden``)
  \item Valärenden
  \item Interpellationer
  \item Propositioner
  \item Motioner
  \item Övriga frågor
  \item Nästa möte
  \item Mötets högtidliga avslutande
\end{enumerate}

\section{Förtjänsttecken och ordnar}

\subsection{Hedersdeltat}

Sektionens finaste förtjänsttecken heter Hedersdeltat och utgörs av en nål med ett delta inramat av en eklövskrans.

\subsubsection{Syfte}

Hedersdeltat utdelas till de sektionsmedlemmar som synnerligen förtjänstfullt verkat ideellt för sektionen.

\subsubsection{Förslagslämning}

Sektionsmedlem kan när som helst inlämna förslag på mottagare av Hedersdeltat, med motivering, till D-rektoratet.

\subsubsection{Utdelning}

D-rektoratet utnämner mottagare av Hedersdeltat, vilka presenteras vid Revisions-SM. Utdelning av förtjänsttecknen sker på Vårbalen eller motsvarande högtidligt tillfälle samma år.

\subsection{Ordnar}

Sektionen har fyra ordnar benämnda ``Klubbmästare Emeritus``,
``Konglig Öfverdrif Emeritus``, ``Storasyskon Emeritus`` och ``Ordförande Emeritus``.

\subsubsection{Ordförande Emeritus}

Ordförande Emeritus tilldelas de sektionsordförande som förtjänstfullt arbetat under en hel mandatperiod.

Vidare gäller att Ordförande Emeriti

\begin{itemize}
  \item erhåller evigt kostnadsfritt medlemskap i sektionen som Alumnimedlem
  \item erhåller årlig speciell inbjudan till Vårbalen.
\end{itemize}

\subsubsection{Klubbmästare Emeritus}

Klubbmästare Emeritus tilldelas de Klubbmästare som förtjänstfullt arbetat under en hel mandatperiod.

\subsubsection{Konglig Öfverdrif Emeritus}

Konglig Öfverdrif Emeritus tilldelas de Konglig Öfverdrif som förtjänstfullt arbetat under en hel mandatperiod.

\subsubsection{Storasyskon Emeritus}

Storasyskon Emeritus tilldelas de Storasyskon som förtjänstfullt arbetat under en hel mandatperiod.

\subsection{Funktionärsmedalj}

\subsubsection{Syfte}

Funktionärsmedaljen utdelas till de sektionsmedlemmar som förtjänstfullt under en hel mandatperiod tjänstgjort som funktionär på sektionen.

\subsubsection{Utdelning}

Endast en medalj per person och år, oavsett antal funktionärsposter personen besuttit. D-rektoratet ansvarar för att medaljen utdelas på Vårbalen eller motsvarande högtidligt tillfälle samma år.

\section{Sektionslokalen}

Sektionslokalen kan endast bokas/hyras av sektionernas styrelser, nämnder och funktionärer, såväl som av organ inom THS, kårföreningar och andra sektioner. Beslut om att bevilja eller avslå bokningsbegäran fattas av sektionslokalsansvariga från fall till fall.

\subsection{Prioritering}

Vid krockande bokningar gäller följande prioritetslista i fallande ordning. Dock kan
ingen tvingas att ändra sin bokning med mindre än två veckors varsel. Inom en prioritetsnivå gäller först till kvarn.

\begin{enumerate}
  \item D-rektoratet/Medietekniks styrelse
  \item Sektionsnämnd eller -funktionär
  \item THS-organ
  \item Kårförening eller annan sektion vid THS.
\end{enumerate}

\subsection{Alkohol}

Det är inte tillåtet att medföra egen alkohol till lokalen, vare sig för försäljning eller enskilt bruk. Om alkoholservering önskas måste en behörig ansvarig från sektionerna närvara. Sektionen förbehåller sig alltid rätten att neka alkoholservering.

\subsection{Övriga regler}

Det får max befinna sig 150 personer i lokalen. Ingen sektionsmedlem kan
vägras tillträde till lokalen, även om den är bokad. Dock skall medlemmar alltid visa största möjliga hänsyn mot den/de som bokat lokalen. Den person som hyr sektionslokalen är personligen ansvarig för de aktiviteter som förekommer där under uthyrningen. I övrigt gäller KTH-handbokens regler för
fester och sammankomster i KTH:s lokaler.

\subsection{Undantag}

Sektionslokalsansvariga kan besluta om undantag från dessa regler, i den mån det är förenligt med THS och KTH:s regler samt svensk lagstiftning, om särskilda skäl föreligger.

\subsection{Stängning}

Sektionslokalen får stängas under en bestämd tidsperiod efter likalydande beslut av respektive Sektionsordförande. Stängning måste dock annonseras i sektionernas officiella informationskanaler senast 4 läsdagar i förväg och beslut om stängning skall alltid prövas på nästföljande styrelsemöte.

\section{Övrigt}

\subsection{Visdomsord}

Det var bättre förr.

\subsubsection{Mer visdomsord}

Och ju förr desto bättre.

\end{document}
