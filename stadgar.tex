\section{Stadgar}

\subsection{§1 Allmänt}

\subsubsection{§1.1 Namn}

Föreningens namn är Konglig Datasektionen, nedan benämnd sektionen.

\subsubsection{§1.2 Ändamål}

Sektionen har till ändamål att främja sina medlemmars studier och vad som har sammanhang med dessa.

Sektionens uppgifter är

\begin{itemize}
  \item att utveckla och upprätthålla kamratskap och sammanhållning bland sektionens medlemmar
  \item att skapa och upprätthålla goda kontakter med närstående personer och organisationer
  \item att aktivt motverka diskriminering inom sektionen.
\end{itemize}

\subsubsection{§1.3 Säte}

Sektionen har sitt säte i Stockholm.

\subsubsection{§1.4 Kårtillhörighet}

Sektionen tillhör Tekniska Högskolans Studentkår, THS.

\subsubsection{§1.5 Verksamhetsår}

Sektionens verksamhetsår löper från 1 januari till 31 december.

\subsubsection{§1.6 Styrdokument}

\paragraph{§1.6.1 Tillgänglighet}

Gällande stadgar och andra styrdokument skall finnas tillgängliga för samtliga sektionsmedlemmar på sektionens officiella webbplats.

\paragraph{§1.6.2 Tolkning}

Skulle tveksamhet uppstå angående dessa stadgar, tolkas dessa av SM. Mellan SM tolkas stadgarna, i stigande företrädesordning, av D-rektoratet och revisorerna. Sådan tolkning skall dock alltid prövas på nästföljande SM.

\paragraph{§1.6.3 Konflikt}

Vid konflikt mellan dessa stadgar eller andra av sektionens styrdokument och THS stadgar eller andra styrdokument skall THS föreskrifter äga företräde. Om sådan konflikt uppmärksammas åligger det D-rektoratet att snarast inkomma till SM med proposition om nödvändiga ändringar.

\paragraph{§1.6.4 Stadgar}

\subparagraph{§1.6.4.1 Stadgeändring}

Stadgarna ändras genom likalydande beslut på två på varandra följande SM, varav minst det ena måste vara ett ordinarie enligt \href{#sammantraden}{§3.9.1}. D-rektoratet skall föra upp fråga om andra läsning av vilande stadgeändring på kallelse och föredragningslista till nästa SM.

\subparagraph{§1.6.4.2 Dispens från stadgarna}

Dispens från bestämmelse i dessa stadgar kan beviljas om SM enhälligt beslutar så. Dispens får dock ej medges från \href{#namn}{§1.1}--\href{#styrdokument}{§1.6} och ej heller om det är till nackdel för enskild sektionsmedlem. Beslut om dispens skall motiveras i mötesprotokollet.

\paragraph{§1.6.5 Reglemente}

\subparagraph{§1.6.5.1 Ändamål}

Reglementet reglerar sektionsverksamheten där stadgarna ej är tillräckligt utförliga. Reglementet är alltid underordnat stadgarna.

\subparagraph{§1.6.5.2 Reglementesändring}

Reglementet kan ändras genom beslut på ett SM. Punkter som måste finnas enligt stadgarna kan dock inte avskaffas utan stadgeändring.

\subsubsection{§1.7 Beslutsnivåer}

Beslut fattas med enkel majoritet om inget annat är föreskrivet.

\subsubsection{§1.8 Officiella informationskanaler}

Information som anslås skall sättas upp i minst A4-format på väl synlig plats i sektionslokalen, samt göras tillgänglig för samtliga sektionsmedlemmar på sektionens officiella webbplats.

\subsubsection{§1.9 Definitioner}

\paragraph{§1.9.1 Läsdag}

Definitionen av läsdagar som benämns i sektionens stadgar, reglemente och praxis är alla dagar som man teoretiskt kan ha något schemalagt under terminstid. Detta inkluderar då läsperiod och tentamensperiod. Dock ej lördagar under tentamensperiod.

\subsubsection{§1.10 Firmatecknare}

Ordförande och kassör tecknar firman var för sig. D-rektoratet kan fatta beslut om ytterligare firmatecknare. Ett sådant beslut ska meddelas på nästa sektionsmöte.

\subsection{§2 Medlemskap}

Sektionsmedlem är

\begin{itemize}
  \item ordinarie sektionsmedlem enligt \href{#ordinarie_sektionsmedlem}{§2.1}
  \item hedersmedlem enligt \href{#hedersmedlem}{§2.2}
  \item stödmedlem enligt \href{#stodmedlem}{§2.3}
  \item juniormedlem enligt \href{#juniormedlem}{§2.4}
\end{itemize}

\subsubsection{§2.1 Ordinarie sektionsmedlem}

Ordinarie sektionsmedlem är medlem i THS som enligt THS föreskrifter eller beslut tillhör sektionen.

\paragraph{§2.1.1 Rättigheter}

Ordinarie sektionsmedlem har rätt att

\begin{itemize}
  \item deltaga med yttranderätt och rösträtt på SM
  \item få motion eller interpellation behandlad av SM
  \item kandidera till samtliga förtroendeuppdrag inom sektionen
  \item närvara på DM såvida det inte beslutats om lyckta dörrar.
\end{itemize}

\subsubsection{§2.2 Hedersmedlem}

Sektionen kan utse till hedersmedlem sådan person som synnerligen främjat sektionens intressen och strävanden. Förslag till hedersmedlem lämnas av sektionsmedlem. Hedersmedlem utses av SM med 5/6 majoritet. Faller fråga om val av hedersmedlem införs varken förslag eller beslut i protokoll.

\subsubsection{§2.3 Stödmedlem}

Stödmedlem är medlem som har anknytning till sektionen och erlagt av kårfullmäktige för stödmedlemskap fastställd avgift. Stödmedlem har närvarorätt på SM. Medlemskap för stödmedlem upphör när stödmedlemmen ej erlagt avgift för stödmedlemskap för angiven period eller då denne hos THS frånsäger sig sitt medlemskap.

\subsubsection{§2.4 Juniormedlem}

Endast för nyligen antagen till Datateknikprogrammet. Medlemskapet gäller från uppropet och två månader framåt. Juniormedlem har närvarorätt och yttranderätt på SM.

\subsubsection{§2.5 Medlemsavgift}

Medlemsavgift för ordinarie medlemmar och alumnimedlemmar fastställs av THS centralt. Hedersmedlemmar är befriade från medlemsavgift.

\subsection{§3 Sektionsmötet}

\subsubsection{§3.1 Ändamål}

Sektionsmötet, SM, är sektionens högsta beslutande organ.

\subsubsection{§3.2 Sammansättning}

Vid SM har samtliga ordinarie sektionsmedlemmar närvarorätt, yttranderätt, yrkanderätt och rösträtt. Sektionens revisorer enligt \href{#revisorer}{§6.1} har närvaro-, yttrande- och yrkanderätt. Hedersmedlemmar, alumnimedlemmar och ledamöter i THS styrelse har närvaro- och yttranderätt. Dessutom kan SM adjungera utomstående med närvarorätt och eventuellt även yttranderätt.

\subsubsection{§3.3 Uppgifter}

Det åligger SM

\begin{itemize}
  \item att fastställa riktlinjer och budget för sektionens verksamhet
  \item att granska funktionärers och revisorers berättelser samt sektionens ekonomiska redovisning
  \item att ta ställning till ansvarsfrihet för D-rektoratet
  \item att om sektionsmedlem så önskar granska protokoll från DM
  \item att välja funktionärer, med undantag av de som väljs vid urnval
  \item att genomföra fyllnadsval vid behov, även till poster som vanligen väljs vid urnval
\end{itemize}

\subsubsection{§3.4 Kallelse}

D-rektoratet kallar till ordinarie och extra SM.

Kallelse till ordinarie SM skall anslås enligt \href{#officiella_informationskanaler}{§1.8} samt tillsändas THS styrelse och THS sakrevisorer senast 15 läsdagar före mötet för att mötet skall anses vara behörigt utlyst.

Kallelse till extra SM skall anslås enligt \href{#officiella_informationskanaler}{§1.8} samt tillsändas THS styrelse och THS sakrevisorer senast 8 läsdagar före mötet för att mötet skall anses vara behörigt utlyst.

Föredragningslista och övriga handlingar skall anslås jämte kallelse senast
5 läsdagar före mötet.

Om minst 30 sektionsmedlemmar, sektionsrevisor enligt \href{#revisorer}{§6.1} eller THS styrelse så begär hos D-rektoratet, skall extra SM hållas inom 20 läsdagar.

Kallelse skall innehålla information om en reservtid och lokal där mötet återupptas om ajournering beslutas enligt \href{#ajournering}{§3.10}. Reservtiden skall vara senast fem (5) läsdagar efter den ordinarie tiden för mötet.

\subsubsection{§3.5 Beslutsmässighet}

SM är beslutsmässigt när mötet är behörigt utlyst enligt \href{#kallelse}{§3.4} och minst 10 sektionsmedlemmar är närvarande.

\subsubsection{§3.6 Beslut}

Beslut kan endast fattas i fråga som antingen enligt stadgarna skall behandlas eller berörs av proposition eller motion. Beslut fattas med enkel majoritet såvida inget annat stadgats. Vid lika röstetal har mötesordföranden utslagsröst, förutom vid personval då lotten avgör. Sluten omröstning skall ske om någon röstberättigad deltagare så begär. SM fattar beslut med parlamentarisk voteringsmetod. Sektionsmedlem äger inte rätt att rösta genom ombud, utan endast genom personlig närvaro på SM.

Ingen får delta i beslut eller leda sammanträdet när frågan om ansvarsfrihet för honom/henne själv behandlas.

\paragraph{§3.6.1 Reservation}

Varje röstberättigad deltagare på SM kan reservera sig mot fattat beslut. Reservation anmäls i samband med beslutet, och lämnas in skriftligen till mötessekreteraren senast 24 timmar efter mötets avslutande. Reservationer skall föras in i protokollet.

\subsubsection{§3.7 Protokoll}

Vid SM skall diskussionsprotokoll föras av mötessekreterare och justeras av mötesordföranden jämte två av mötet utsedda justerare. Protokoll skall innehålla en förteckning över närvarande, röstberättigade medlemmar. Protokoll skall i justerat skick anslås enligt \href{#officiella_informationskanaler}{§1.8} samt tillsändas THS styrelse inom 14 dagar.

\subsubsection{§3.8 Interpellation, motion och proposition}

Motion eller interpellation till SM skall vara D-rektoratet tillhanda senast 10 läsdagar före SM. D-rektoratet skall skriftligen besvara samtliga motioner.

Förslag från D-rektoratet benämns proposition.

Interpellation skall besvaras skriftligen av den funktionär den ställts till, eller av D-rektoratet om den ställts till dem.

D-rektoratet ansvarar för att motioner, interpellationer, svar på dessa samt propositioner anslås tillsammans med föredragningslistan.

\subsubsection{§3.9 Sammanträden}

Det skall förflyta minst fem läsdagar mellan två på varandra följande SM. SM får inte hållas under tentamensperiod eller ferie.

\paragraph{§3.9.1 Ordinarie SM}

Som ordinarie SM räknas de SM som regleras i reglementet, samt övriga, av D-rektoratet, utlysta ordinarie SM. Som ordinarie SM räknas ej extra SM. Det skall hållas minst ett ordinarie SM per termin.

\paragraph{§3.9.2 Extra SM}

D-rektoratet kan, själv eller på anmodan, kalla till extra SM. Extra SM kan endast behandla den eller de frågor som angivits i kallelsen, således behandlas ej övriga motioner, interpellationer eller propositioner. Dock kan övrig fråga väckas.

\subsubsection{§3.10 Ajournering}

Mötets ordförande äger rätt att ajournera mötet till den, enligt \href{#kallelse}{§3.4}, i kallelsen angivna reservtiden.

\subsection{§4 D-rektoratet}

\subsubsection{§4.1 Ändamål}

D-rektoratet är sektionens styrelse och högsta verkställande organ.

\subsubsection{§4.2 Sammansättning}

D-rektoratet består av

\begin{itemize}
  \item Sektionsordförande
  \item Vice Sektionsordförande
  \item Kassör
  \item Sekreterare
  \item Ledamot för sociala frågor och relationer
  \item Ledamot för studiemiljöfrågor
  \item Ledamot för utbildningsfrågor
\end{itemize}

Dessa har närvaro-, yttrande-, yrkande- och rösträtt vid DM. Sektionens revisorer enligt §6.1 har närvaro-, yttrande- och yrkanderätt vid DM. Funktionärer har närvaro- och yttranderätt vid DM. Övriga sektionsmedlemmar har närvarorätt vid DM. Därutöver äger D-rektoratet rätt att adjungera person med närvaro- eller närvaro- och yttranderätt för viss fråga eller helt möte. D-rektoratet äger vidare, om synnerliga skäl föreligger, rätt att besluta om lyckta dörrar, vilket utestänger samtliga utan yrkanderätt.

\subsubsection{§4.3 D-rektoratsmöte}

\paragraph{§4.3.1 Kallelse}

Sektionsordförande kallar till D-rektoratsmöte, DM. Kallelsen skall anslås enligt \href{#officiella_informationskanaler}{§1.8} samt skickas med e-post till D-rektoratets ledamöter och sektionens funktionärer senast 5 läsdagar före mötet.

\paragraph{§4.3.2 Beslut}

DM är beslutsmässigt om minst hälften av dess ledamöter är närvarande, och mötet är behörigt utlyst enligt \href{#d-rektoratsmote}{§4.3.1}. Vid lika röstetal har mötesordförande utslagsröst.

\paragraph{§4.3.3 Protokoll}

På DM skall protokoll föras. Protokollet skall justeras av mötesordföranden jämte en av mötet utsedd justerare. Protokollet skall anslås enligt \href{#officiella_informationskanaler}{§1.8} i justerat skick senast 14 dagar efter mötet.

\subsubsection{§4.4 Uppgifter}

Det åligger D-rektoratet

\begin{itemize}
  \item att sköta sektionens löpande förvaltning
  \item att verkställa av SM fattade beslut
  \item att i brådskande fall utöva SM:s befogenheter. Sådant fall skall dock alltid prövas på nästkommande SM
  \item att efter skriftlig begäran från en av SM vald funktionär entlediga densamme
  \item att vid behov och efter majoritetsbeslut vid DM tillförordna intresserad sektionsmedlem till vakant post inom sektionen. D-rektoratet får dock ej tillförordna D-rektoratsledamot, sektionsrevisor, ledamot eller suppleant i kårfullmäktige eller valberedare
  \item att vid behov och efter majoritetsbeslut vid DM utöva ordförandeskap för nämnd i dess ordförandes ställe.
  \item att svara för att verksamhetsplan, budget, verksamhetsberättelse och årsbokslut upprättas
  \item att, om så anses nödvändigt, avsätta en av sektionen vald funktionär, dock ej styrelseledamot, revisor, kårfullmäktigesuppleant, kårfullmäktigeledamot eller valberedare. En sådan avsättning skall dock alltid prövas på nästkommande SM.
\end{itemize}

\subsubsection{§4.5 Brådskande ärenden}

I brådskande fall äger D-rektoratet rätt att utöva SM:s befogenheter. D-rektoratet äger dock ej därigenom rätt att ändra stadgar eller reglemente eller att bevilja dispens från stadgarna. Beslut enligt detta stycke skall prövas på nästföljande SM.

I brådskande fall äger sektionsordförande rätt att utöva D-rektoratets befogenheter. Sektionsordförande äger dock ej därigenom rätt att utöva SM:s befogenheter enligt första stycket eller att besluta om utgifter överstigande 2000 SEK. Beslut enligt detta stycke skall prövas på nästföljande DM.

\subsubsection{§4.6 Ställföreträdande sektionsordförande}

Om sektionsordförande är oförmögen att göra så, utövar vice sektionsordförande dennes befogenheter, och fullgör dennes plikter.

\subsubsection{§4.7 Per capsulam-beslut}

Vid per capsulam beslut gäller 2/3-majoritet och att beslut prövas på nästkommande DM.

\subsubsection{§4.8 D-rektiv}

D-rektoratet må, om det så önskar, utfärda D-rektiv, vilka utgöra rekommendationer å de enskilda sektionsmedlemmarnas liv och leverne.

\subsection{§5 Organisation}

\subsubsection{§5.1 Nämnder}

\paragraph{§5.1.1 Ändamål}

En nämnd är ett officiellt sektionsorgan med syfte att ansvara för en viss del av sektionens verksamhet. Nämnder driver sin verksamhet självständigt inom ramen för av SM och D-rektoratet fattade beslut. Nämnder är de, som upptas i reglementet.

\paragraph{§5.1.2 Sammansättning och verksamhet}

En nämnds sammansättning och verksamhet regleras i reglementet.

\subparagraph{§5.1.2.1 Ordförande}

För varje nämnd skall det finnas en eller flera funktionärer som är ordförande. Nämndens ordförande är ansvarig för nämndens verksamhet samt att dess reglemente hålls aktuellt.

\paragraph{§5.1.3 Skyldigheter}

Nämnd är skyldig att upprätta verksamhetsberättelse, samt även annars på anmodan från D-rektoratet eller SM fullständigt redovisa sin verksamhet för densamme.

\paragraph{§5.1.4 Obligatoriska nämnder}

Det skall finnas en studienämnd, en sektionslokalnämnd samt en mottagningsnämnd.

\subsubsection{§5.2 Funktionärer}

\paragraph{§5.2.1 Ändamål}

Funktionär är den som av SM eller vid urnval har valts till ett förtroendeuppdrag. En funktionärs verksamhet och uppdrag regleras i reglementet.

\paragraph{§5.2.2 Skyldigheter}

Funktionär ansvarar för sitt verksamhetsområde samt för att funktionärens del av reglementet hålls aktuellt. Funktionär är skyldig att löpande hålla D-rektoratet informerat om sitt verksamhetsområde, samt att på anmodan från D-rektoratet eller SM fullständigt redovisa sin verksamhet för densamme.

\paragraph{§5.2.3 Mandatperiod}

Funktionärs mandatperiod sammanfaller med verksamhetsår om inget annat är föreskrivet i reglementet. Ordinarie val skall hållas på mandatperiodens sista ordinarie SM.

\paragraph{§5.2.4 Valberedning}

SM skall utse en eller flera valberedare. Om flera utses, skall en väljas till
sammankallande. Endast sammankallande eller ensam valberedare betraktas som funktionär.

Valberedningen skall enligt \href{#officiella_informationskanaler}{§1.8} anslå en nomineringslista senast 15 läsdagar före SM då ordinarie val sker. På denna lista kan sektionsmedlemmar nominera funktionärer.

Nominering till funktionärspost måste lämnas in senast 5 läsdagar före det SM där valet sker. Nominering till funktionärspost måste accepteras senast en läsdag före det SM där valet sker för att kandidaturen ska vara giltig.

Valberedningen skall tillfråga de nominerade och i anslutning till handlingarna för SM då val sker anslå en ny lista på samtliga nominerade som tackat ja till nomineringen.

\subparagraph{§5.2.4.1 Urnval}

Sektionsordförande, vice sektionsordförande och övriga ledamöter och suppleanter till THS Kårfullmäktige väljs med urnval i enlighet med sektionens reglemente och THS stadgar.

\paragraph{§5.2.5 Obligatoriska funktionärer}

Utöver D-rektoratets ledamöter, valberedare, revisorer och ordförande för de under \href{#namnder}{§5.1.4} uppräknade nämnderna, som regleras särskilt, skall det finnas en programansvarig student och ett studerandeskyddsombud.

\subsection{§6 Revision}

\subsubsection{§6.1 Revisorer}

SM skall utse två revisorer. Tillsammans med av THS därtill utsedda revisorer utgör dessa sektionens revisorer. Om inte THS beslutar annorlunda skall revisionsberättelsen undertecknas av minst två av dessa.

\paragraph{§6.1.1 Befogenheter}

Revisorerna har rätt

\begin{itemize}
  \item att närhelst de så önskar ta del av samtliga räkenskaper, protokoll och andra handlingar
  \item att begära och erhålla upplysningar rörande verksamhet och förvaltning
  \item att bevaka samtliga sektionsorgans sammanträden med yttrande och yrkanderätt
  \item att kalla till möte med samtliga sektionsorgan.
\end{itemize}

\paragraph{§6.1.2 Uppgifter}

Det åligger revisorerna

\begin{itemize}
  \item att fortlöpande granska sektionens förvaltning och verksamhet
  \item att senast 5 läsdagar före de SM vid vilka fråga om ansvarsfrihet behandlas anslå revisionberättelse enligt \href{#officiella_informationskanaler}{§1.8} samt inlämna revisionberättelse till D-rektoratet.
\end{itemize}

\subsubsection{§6.2 Verksamhetsberättelse och årsbokslut}

Sektionens verksamhetsberättelse och årsbokslut skall överlämnas till revisorerna senast 15 läsdagar före det SM på vilka de skall granskas, samt anslås enligt \href{#officiella_informationskanaler}{§1.8} senast 10 läsdagar före samma SM.


