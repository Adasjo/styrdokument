\documentclass{dgovdoc}

\usepackage[swedish]{babel}
\usepackage[T1]{fontenc}
\usepackage[utf8]{inputenc}

\usepackage{hyperref}

\title{Alkoholpolicy}

\begin{document}

\maketitle

\section{Relation till andra föreskrifter}

THS har en alkohol- och drogpolicy som fastställs av Kårfullmäktige och gäller
inom hela THS organisation, Datasektionen inkluderad. Detta dokument ska ses
som ett tillägg till THS alkohol- och drogpolicy.

I de fall detta dokument står i konflikt med Sveriges rikes lag, THS alkohol-
och drogpolicy, eller andra av KTH eller THS uppsatta regler är detta dokument
underordnat.

\section{Omfattning}

Denna policy omfattar samtliga aktiviteter som genomförs i sektionens regi
eller i sektionslokalen, och skall efterlevas.

\section{Öppnande av bar}

Endast Klubbmästaren och Sektionsordförande har rätt att öppna baren. Som följd
av detta måste Sektionsordförande eller Klubbmästare skriva under en
festanmälan för arrangemang där alkohol serveras.

\section{Försäljning av dryck på kredit (strecklistor)}

Försäljning av dryck på kredit får endast ske då avtal upprättas mellan
sektionen och personen som tar emot drycken. Sådana skulder skall regleras
regelbundet.

\section{Alkohol under skoltid}

08:00--17:00 varje helgfri vardag är att betrakta som skoltid, och under denna
tid får alkohol ej förtäras i sektionslokalen om det ej är som del i ett
arrangemang för vilken lokalen är bokad.

\section{Alkohol utanför sektionslokalen}

Det är inte tillåtet att medföra sektionens alkohol ut från serveringsytan,
även om baren är stängd, i annat ändamål än att flytta alkoholen till ett av
sektionen genomfört arrangemang eller för att lämna tillbaka alkoholen till
återförsäljare.

\section{Nykter personal}

Personal som arbetar på sektionens arrangemang skall alltid vara nykter,
oavsett sammanhang. Serverings- samt festansvarig skall alltid vara helt
alkoholfri under arrangemang.

\section{Arrangemang utanför det permanenta serveringstillståndet}

Vid arrangemang utanför sektionens permanenta serveringstillstånd skall
nödvändiga tillstånd skaffas. Om tillfälligt alkoholtillstånd ej erhållits
skall alkoholservering likväl endast ske i enlighet med alkohollagen. Detta
innebär exempelvis att alkoholförsäljning med vinstintresse på Osqvik endast
får ske om tillstånd erhållits.

\section{Alkohol bakom bardisk}

Alkohol får inte förtäras bakom baren.

\section{Självservering}

Ingen får sälja och servera alkohol till sig själv.

\section{Alkohol bekostad av sektionen}

Högst 1/3 av pengarna spenderade vid ett tillfälle med en intern grupp får användas för att bekosta alkohol. Med detta menas att pengar under t.ex. en teambuilding läggs för att subventionera alkohol. 

\end{document}
