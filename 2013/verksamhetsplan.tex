\documentclass{dgovdoc}

\usepackage[swedish]{babel}
\usepackage[T1]{fontenc}

\usepackage{hyperref}

\title{Verksamhetsplan 2013}

\begin{document}

\maketitle

\section{D-rektoratet}

\subsection{Kårmedlemskap}

D-rektoratet skall

\begin{itemize}
\item synliggöra THS:s och sektionens arbete i samarbete med Informationsorganet.
\item öka antalet medlemmar för att behålla sektionens trovärdighet gentemot
  skolan, målet är att över 75\% av de studerande vid datateknikprogrammet på CSC
  skall vara medlemmar till årsskiftet 2013/2014.
\end{itemize}

\subsection{Löpande arbete}

D-rektoratet skall

\begin{itemize} 
\item minst två gånger i månaden genomföra arbetsmöten där aktivt
  styrelsearbete sker gemensamt.
\item lämna återkoppling på frågor och funderingar från medlemmarna inom en
  arbetsdag.
\item ha D-rektoratsmöte minst två gånger per termin.
\end{itemize}

\subsection{Ekonomi}

D-rektoratet skall

\begin{itemize}
\item sköta den löpande bokföringen med hjälp av kassören.
\end{itemize}

\subsection{Andra sektioner och systerprogram}

Sektionen har ett nära samarbete med Sektionen för Medieteknik, ett samarbete
kallat Team CSC. Inom ramen för detta samarbete har styrelserna från respektive
sektioner drivit gemensamma frågor gentemot skolan och inom THS. Detta
samarbete har gett gott resultat i de frågor som drivits.

D-rektoratet skall

\begin{itemize}
\item arbeta för att behålla och vidareutveckla Team CSC.
\item aktivt arbeta för att hitta frågor som kan drivas gemensamt av flera
  sektioner och ta initiativ till sådana samarbeten över sektionsgränserna.
\item verka för att ha ett nära samarbete med andra sektioners styrelser.
\item besöka andra sektioners styrelser och diskutera aktuella händelser på
  KTH. Detta görs med fördel under avslappnade former.
\end{itemize}

\subsection{Ökat sektionsengagemang}

D-rektoratet skall

\begin{itemize}
\item finna tillfällen för nämnderna att synas utöver mottagningen.
\end{itemize}

\subsection{Övrigt}

D-rektoratet skall

\begin{itemize}
\item anordna minst fyra D-funkmiddagar för att öka kommunikation mellan
  funktionärer och nämnder.
\item anordna en Skiftesgasque för att tacka sektionsaktiva medlemmar.
\item anordna två styrelseöverlämningar, en efter varje SM som enligt stadgarna
  ska behandla val av styrelseledamöter.
\end{itemize}

\section{Fenixorden}

Fenixorden skall

\begin{itemize} 
\item alltid främja skapandet av nya nämnder. 
\item dela ut en medalj till en lämplig kandidat.
\end{itemize}

\section{Idrottsnämnden}

Idrottsnämnden skall

\begin{itemize}
\item anordna regelbundna träningstillfällen och sportevenemang för sektionens
  medlemmar.
\item samordna sektionens deltagande i relevanta sporttävlingar.
\item annonsera och synliggöra nämndens aktiviteter för sektionens medlemmar.
\end{itemize}

\section{Informationsorganet}

Informationsorganet skall

\begin{itemize}
\item hjälpa D-rektoratet och övriga nämnder med informationsspridning.
\item underhålla och vidareutveckla sektionens webbplats.
\item ansvara för drift och underhåll av sektionens datorresurser.
\item ha en samordnande roll för sektionens informationsflöden.
\item ha regelbunden kontakt med sektionens nämnder i syfte att höja kvalitén
  på informationsspridningen.
\item hålla en serie grafikrelaterade workshops.
\item främja grafisk medvetenhet på sektionen.
\end{itemize}

\section{Internationella utskottet}

Internationella utskottet skall

\begin{itemize}
\item samordna sektionens internationella verksamhet. Detta inkluderar att
  hålla kontakten med ISS ordförande på THS och ansvariga för utbytesstudier på
  CSC:s kansli och institutioner.
\item hålla sektionsmedlemmarna informerade om den internationella verksamheten
  på sektionen.
\item hjälpa till i THS centrala mottagningsverksamhet för utländska studenter.
  Detta inkluderar att rekrytera faddrar och koordinera fadderverksamheten.
\item fungera som kontaktyta för utländska studenter på sektionen.
\item expandera sin verksamhet och rekrytera medlemmar till nämnden.
\end{itemize}

\section{Jämlikhetsnämnden}

Jämlikhetsnämnden skall

\begin{itemize}
\item genomföra minst en undersökning om hur jämlikhetssituationen på sektionen
  är.
\item utvärdera resultaten av undersökningen och presentera dessa för
  D-rektoratet och sektionens medlemmar.
\item upprätta en tydlig kontakt dit medlemmar kan vända sig med
  jämlikhetsfrågor.
\item verka för att utöka ett samarbete mellan jämlikhetsnämnden och
  jämlikhetsorgan på CSC och THS.
\item upprätta och underhålla en handlingsplan, inklusive krishantering, för
  jämlikhetsfrågor.
\end{itemize}

\section{Klubbmästeriet}

Klubbmästeriet skall

\begin{itemize}
\item bokföra sin verksamhet regelbundet.
\item arrangera onsdagspubar.
\item arrangera Plums.
\item arrangera minst fyra fester för sektionens medlemmar.
\item anordna tentapub efter varannan tentaperiod.
\item bistå med råd/erfarenhet när andra nämnder på sektionen arrangerar stora
  evenemang.
\end{itemize}

\section{METAdorerna}

METAdorerna skall

\begin{itemize}
\item regelbundet genomföra underhåll av de maskiner och delar av
  sektionslokalen som kräver löpande underhåll och tillsyn.
\item regelbundet kontrollera att alla inventarier i sektionslokalen är hela
  och rena samt ersätta, reparera eller ta bort trasiga inventarier.
\item tillse att sektionslokalen städas regelbundet.
\item anordna en tackfest en gång per läsår för dem som hjälpt till att städa
  sektionslokalen.
\item ha ett samarbete med motsvarande nämnd på Sektionen för Medieteknik.
\end{itemize}

\section{Mottagningen}

Mottagningen skall

\begin{itemize}
\item hålla en mottagning för de nya studenterna som börjar till hösten med
  syftet att få dem att känna sig välkomna och introducera dem till KTH, THS och
  sektionen.
\item eftersträva att få nØllan att vilja bli medlemmar i sektionen och THS.
\item informera nØllan om THS centralts funktion och verksamhet.
\item fungera som en naturlig inkörsport till sektionsengagemang och socialt
  umgänge med andra studenter.
\item ta in feedback från nØllan och sammanställa informationen skriftligt i
  syftet att förbättra mottagningen från nØllans synvinkel.
\end{itemize}

\section{Näringslivsgruppen}

Näringslivsgruppen skall

\begin{itemize}
\item arrangera D-dagen.
\item administrera lunchföreläsningar.
\item administrera företagspubar.
\item tillhandahålla annonseringsmöjlighet för företag.
\end{itemize}

\section{Prylmånglaren}

Prylmånglaren skall

\begin{itemize}
\item söka sponsring till overaller.
\item utfodra datasektionens medlemmar med märken och dylikt.
\item hålla lämpligt antal försäljningstillfällen.
\item ta fram nya prylar. Men även fixa fram så att det finns av gamla.
\item vara till hands för alla sektionsmedlemmar.
\item se till att overallerna är framme till första möjliga tillfälle för
  Ettan.
\item ska vara behjälplig vid framtagandet årets årskursmärke.
\end{itemize}

\section{Qulturnämnden}

Qulturnämnden skall

\begin{itemize}
\item anordna regelbundna qulturella aktiviteter för sektionsmedlemmar såsom
  spelkvällar och filmvisningar.
\item sträva efter att med jämna mellanrum arrangera qulturella event utanför
  sektionslokalen.
\item aktivt och kontinuerligt utveckla och anpassa sin verksamhet i linje med
  sektionsmedlemmarnas intressen och efterfrågan.
\item ha ett löpande samarbete med motsvarande nämnd inom Sektionen för
  Medieteknik.
\end{itemize}

\section{Redaqtionen}

Redaqtionen skall

\begin{itemize}
\item ge ut en nØlledBuggen under mottagningen.
\item träffas till och från för att arbeta fram en ny dBuggen.
\item ge ut minst en dBuggen utöver nØlledBuggen.
\end{itemize}

\section{Spexmästeriet}

Spexmästeriet skall verka för att

\begin{itemize}
\item ha roligt.
\item se till att andra har roligt ibland också.
\item ``vinna'' Sångartäfvlan på THS.
\item synas.
\item se roliga ut.
\item ge ut ett nytt album!
\item gyckla under mottagningen!
\end{itemize}

\end{document}
